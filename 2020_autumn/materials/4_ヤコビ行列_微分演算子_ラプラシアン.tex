\documentclass[dvipdfmx,a4paper,11pt]{article}
\usepackage[utf8]{inputenc}
%\usepackage[dvipdfmx]{hyperref} %リンクを有効にする
\usepackage{url} %同上
\usepackage{amsmath,amssymb} %もちろん
\usepackage{amsfonts,amsthm,mathtools} %もちろん
\usepackage{braket,physics} %あると便利なやつ
\usepackage{bm} %ラプラシアンで使った
\usepackage[top=30truemm,bottom=30truemm,left=25truemm,right=25truemm]{geometry} %余白設定
\usepackage{latexsym} %ごくたまに必要になる
\renewcommand{\kanjifamilydefault}{\gtdefault}
\usepackage{otf} %宗教上の理由でmin10が嫌いなので


\usepackage[all]{xy}
\usepackage{amsthm,amsmath,amssymb,comment}
\usepackage{amsmath}    % \UTF{00E6}\UTF{0095}°\UTF{00E5}\UTF{00AD}\UTF{00A6}\UTF{00E7}\UTF{0094}¨
\usepackage{amssymb}  
\usepackage{color}
\usepackage{amscd}
\usepackage{amsthm}  
\usepackage{wrapfig}
\usepackage{comment}	
\usepackage{graphicx}
\usepackage{setspace}
\setstretch{1.2}


\newcommand{\R}{\mathbb{R}}
\newcommand{\Z}{\mathbb{Z}}
\newcommand{\N}{\mathbb{N}}
\newcommand{\C}{\mathbb{C}} 



   %当然のようにやる.
\allowdisplaybreaks[4]
   %もちろん.
%\title{第1回. 多変数の連続写像 (岩井雅崇, 2020/10/06)}
%\author{岩井雅崇}
%\date{2020/10/06}
%ここまで今回の記事関係ない
\usepackage{tcolorbox}
\tcbuselibrary{breakable, skins, theorems}

\theoremstyle{definition}
\newtheorem{thm}{定理}
\newtheorem{lem}[thm]{補題}
\newtheorem{prop}[thm]{命題}
\newtheorem{cor}[thm]{系}
\newtheorem{claim}[thm]{主張}
\newtheorem{dfn}[thm]{定義}
\newtheorem{rem}[thm]{注意}
\newtheorem{exa}[thm]{例}
\newtheorem{conj}[thm]{予想}
\newtheorem{prob}[thm]{問題}
\newtheorem{rema}[thm]{補足}

\DeclareMathOperator{\Ric}{Ric}
\DeclareMathOperator{\Vol}{Vol}
 \newcommand{\pdrv}[2]{\frac{\partial #1}{\partial #2}}
 \newcommand{\drv}[2]{\frac{d #1}{d#2}}
  \newcommand{\ppdrv}[3]{\frac{\partial^2 #1}{\partial #2 \partial #3}}


%ここから本文.
\begin{document}
%\maketitle
\begin{center}
{\Large 第4回. ヤコビ行列・微分演算子・ラプラシアン \\(川平先生の本, 第20・22章の内容)}
\end{center}

\begin{flushright}
 岩井雅崇, 2020/10/27
\end{flushright}


\section{ヤコビ行列}
\begin{tcolorbox}[
    colback = white,
    colframe = green!35!black,
    fonttitle = \bfseries,
    breakable = true]
    \begin{dfn}
 領域$D$上の$C^1$級変数変換
 $$
\begin{array}{ccccc}
\Phi: &D & \rightarrow & \R^2 & \\
&(u,v) & \longmapsto & (x(u,v),y(u,v))&
\end{array}
$$
について, \underline{$\Phi$のヤコビ行列 }$D\Phi$を次で定める.
$$
D\Phi
=
\left(\begin{array}{cc} \pdrv{x}{u} & \pdrv{x}{v} \\ \pdrv{y}{u}& \pdrv{y}{v} \\ \end{array} \right)
$$

    \end{dfn}
    \end{tcolorbox}

\begin{exa}
\text{}
\begin{itemize}
\item $a,b,c,d$を定数とする.
1次変換
$\Phi(u,v)  = (au+bv, cu+dv)$について, 
$D\Phi = \left(\begin{array}{cc} a& b\\ c& d\\ \end{array} \right).$
\item 極座標変換$\Phi(u,v)  = (u \cos v, u \sin v)$について, 
$D\Phi = \left(\begin{array}{cc} \cos v& -u\sin v\\ \sin v& u \cos v\\ \end{array} \right).$
\end{itemize}

\end{exa}


\begin{tcolorbox}[
    colback = white,
    colframe = green!35!black,
    fonttitle = \bfseries,
    breakable = true]
    \begin{dfn}
$C^1$級変数変換を$(x,y) = \Phi (u,v)$, $(z,w) = \Psi (x,y)$とする.

\begin{enumerate}

\item\underline{合成変換$\Psi \circ \Phi$}を
$(z,w) = \Psi \circ \Phi(u,v)=\Psi(x(u,v),y(u,v))$とする.

\item $C^1$級変数変換を$(x,y) = \Phi (u,v)$が1対1であるとき, ある$C^1$級変数変換$(u,v) = \Omega(x,y)$があって$(u,v) = \Omega\circ\Phi (u,v)$となる.
この$\Omega$を\underline{$\Phi $の逆変換}といい$\Phi^{-1}$とかく.
\end{enumerate}
    \end{dfn}
    \end{tcolorbox}
   \footnote{$C^1$級変数変換を$\Phi $が1対1とは$\Phi (u_1,v_1)=\Phi (u_2,v_2)$ならば$(u_1,v_1)=(u_2,v_2)$となること. この定義においての逆関数の存在は逆関数定理(第7回)によりわかる.}   


\begin{tcolorbox}[
    colback = white,
    colframe = green!35!black,
    fonttitle = \bfseries,
    breakable = true]
    \begin{thm}
$C^1$級変数変換を$(x,y) = \Phi (u,v)$, $(z,w) = \Psi (x,y)$について
その合成変換を
$(z,w) = \Psi \circ \Phi(u,v)$とするとき, 
$$
D (\Psi \circ \Phi)
= D \Psi D\Phi
=
\left(\begin{array}{cc} \pdrv{z}{x} & \pdrv{z}{y} \\ \pdrv{w}{x}& \pdrv{w}{y} \\ \end{array} \right)
\left(\begin{array}{cc} \pdrv{x}{u} & \pdrv{x}{v} \\ \pdrv{y}{u}& \pdrv{y}{v} \\ \end{array} \right).
$$
特に$\Phi $の逆関数が存在するとき, $\det D \Phi \neq 0$ならば
$$
D \Phi^{-1}
= (D\Phi)^{-1}
=
\left(\begin{array}{cc} \pdrv{x}{u} & \pdrv{x}{v} \\ \pdrv{y}{u}& \pdrv{y}{v} \\ \end{array} \right)^{-1}.
$$
    \end{thm}
    \end{tcolorbox}
    
\begin{exa}

     極座標変換$\Phi(u,v)  = (u \cos v, u \sin v)$について,
$D\Phi = \left(\begin{array}{cc} \cos v& -u\sin v\\ \sin v& u \cos v\\ \end{array} \right)$より, 
$$D\Phi^{-1} = \left(\begin{array}{cc} \cos v& -u\sin v\\ \sin v& u \cos v\\ \end{array} \right)^{-1}
= \left(\begin{array}{cc} \cos v& \sin v\\ \frac{-\sin v}{u}& \frac{\cos v}{u}\\ \end{array} \right)
= \left(\begin{array}{cc} \frac{x}{\sqrt{x^2+y^2}}& \frac{y}{\sqrt{x^2+y^2}}\\\ \frac{-y}{x^2+y^2}& \frac{x}{x^2+y^2}\\ \end{array} \right).
$$
    
    \end{exa}
    
\section{$n$階偏導関数・$C^n$級}

\begin{tcolorbox}[
    colback = white,
    colframe = green!35!black,
    fonttitle = \bfseries,
    breakable = true]
    \begin{dfn}
$f(x,y)$を$C^1$級関数とする.
\begin{itemize}
\item $\pdrv{f}{x} , \pdrv{f}{y}$を\underline{$f$の1階偏導関数}という.
\item $\pdrv{f}{x} , \pdrv{f}{y}$が$C^1$級であるとき, これらの導関数
$$
\pdrv{^2f}{x^2} = \pdrv{}{x}\left( \pdrv{f}{x} \right), 
\ppdrv{f}{x}{y} =\pdrv{}{x}\left( \pdrv{f}{y} \right), 
\ppdrv{f}{y}{x} =\pdrv{}{y}\left( \pdrv{f}{x} \right), 
\pdrv{^2f}{y^2} = \pdrv{}{y}\left( \pdrv{f}{y} \right)
$$
を\underline{$f$の2階偏導関数}という.
\item 同様に, $\pdrv{^3f}{x^3} = \pdrv{}{x}\left( \pdrv{^2f}{x^2} \right)$や
$\frac{\partial^3 f}{\partial x \partial y \partial x} = \pdrv{}{x}\left( \ppdrv{f}{y}{x}\right)$などが考えられるが, これらを\underline{$f$の3階偏導関数}という.
$n$階偏導関数も同様である.

\item $n$を正の自然数とする.
\underline{$f(x,y)$が$C^n$級}であるとは$f$の$n$階偏導関数が存在し連続であること.
\item \underline{$f(x,y)$が$C^{\infty}$級}とは, 全ての正の自然数$n$について$f(x,y)$が$C^n$級であること.

\end{itemize}

    \end{dfn}
    \end{tcolorbox}

みんながよく知っている関数は$C^{\infty}$級関数.
($x^2+1, \sin x, \log x, e^x$ などなど... )
\begin{exa}
$f(x,y) = x^2 y^3$. $C^{\infty}$級関数. 偏導関数は以下の通り.

$$
\pdrv{f}{x}=2xy^3,  \pdrv{f}{y}=3x^2y^2.
$$

$$
\pdrv{^2f}{x^2} = 2y^3, 
\ppdrv{f}{x}{y} =6xy^2, 
\ppdrv{f}{y}{x} =6xy^2, 
\pdrv{^2f}{y^2} = 6x^2y.
$$
\end{exa}

\begin{tcolorbox}[
    colback = white,
    colframe = green!35!black,
    fonttitle = \bfseries,
    breakable = true]
    \begin{thm}
$f(x,y)$が$C^2$級関数ならば
$$
\ppdrv{f}{x}{y} =\ppdrv{f}{y}{x}.
$$
特に, $f(x,y)$が$C^\infty$級関数ならば, 自由に偏微分の順序交換ができる.
    \end{thm}
    \end{tcolorbox}

\section{微分演算子・ラプラシアン}

\begin{tcolorbox}[
    colback = white,
    colframe = green!35!black,
    fonttitle = \bfseries,
    breakable = true]
    \begin{dfn}[この授業だけの定義]
$m$を正の自然数とし$a_{ij}(x,y)$を関数として
$$
D = \sum_{j=0}^{m} \sum_{i=0}^{m-j} a_{ij}(x,y) \left( \pdrv{^i}{x^i} \right)\left( \pdrv{^j}{y^{j}} \right)
$$ 
と書ける作用素を\underline{微分演算子}という.

$D$は$C^{\infty}$関数$f$に対して次のように作用する.
$$
Df = \sum_{j=0}^{m} \sum_{i=0}^{m-j} a_{ij}(x,y) {\frac{\partial^{i+j} f}{\partial x^i \partial y^{j}}}
$$
    \end{dfn}
    \end{tcolorbox}
  %$\pdrv{}{x}, x\pdrv{}{x} , \pdrv{^2}{x^2}+\pdrv{^2}{y^2}$などは微分作用素
    
 \begin{exa}
 $D_1=\pdrv{}{x}, D_2=x\pdrv{}{x}$とおく. これらは微分演算子.
 $D_1D_2 = x\left( \pdrv{}{x^2} \right)$だが
  $D_2D_1 =\pdrv{}{x} + x\left( \pdrv{}{x^2} \right)$である.
  特に$D_1D_2 \neq D_2D_1$.
 \end{exa}
 
 \begin{tcolorbox}[
    colback = white,
    colframe = green!35!black,
    fonttitle = \bfseries,
    breakable = true]
    \begin{dfn}
$$
\Delta = \pdrv{^2}{x^2} + \pdrv{^2}{y^2}
$$
と書ける微分演算子を\underline{ラプラシアン}という.
    \end{dfn}
    \end{tcolorbox}

    
 \begin{exa}
極座標変換$(x(u,v), y(u,v))  = (u \cos v, u \sin v)$とする.
このとき, 
$$
\Delta = \pdrv{^2}{x^2} + \pdrv{^2}{y^2} 
= \pdrv{^2}{u^2} + \frac{1}{u}\pdrv{}{u}  +  \frac{1}{u^2}\pdrv{^2}{v^2} 
$$
 \end{exa}
 
\end{document}
