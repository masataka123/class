\documentclass[dvipdfmx,a4paper,11pt]{article}
\usepackage[utf8]{inputenc}
%\usepackage[dvipdfmx]{hyperref} %リンクを有効にする
\usepackage{url} %同上
\usepackage{amsmath,amssymb} %もちろん
\usepackage{amsfonts,amsthm,mathtools} %もちろん
\usepackage{braket,physics} %あると便利なやつ
\usepackage{bm} %ラプラシアンで使った
\usepackage[top=30truemm,bottom=30truemm,left=25truemm,right=25truemm]{geometry} %余白設定
\usepackage{latexsym} %ごくたまに必要になる
\renewcommand{\kanjifamilydefault}{\gtdefault}
\usepackage{otf} %宗教上の理由でmin10が嫌いなので


\usepackage[all]{xy}
\usepackage{amsthm,amsmath,amssymb,comment}
\usepackage{amsmath}    % \UTF{00E6}\UTF{0095}°\UTF{00E5}\UTF{00AD}\UTF{00A6}\UTF{00E7}\UTF{0094}¨
\usepackage{amssymb}  
\usepackage{color}
\usepackage{amscd}
\usepackage{amsthm}  
\usepackage{wrapfig}
\usepackage{comment}	
\usepackage{graphicx}
\usepackage{setspace}
\setstretch{1.2}


\newcommand{\R}{\mathbb{R}}
\newcommand{\Z}{\mathbb{Z}}
\newcommand{\N}{\mathbb{N}}
\newcommand{\C}{\mathbb{C}} 
\newcommand{\Area}{\text{Area}}





   %当然のようにやる.
\allowdisplaybreaks[4]
   %もちろん.
%\title{第1回. 多変数の連続写像 (岩井雅崇, 2020/10/06)}
%\author{岩井雅崇}
%\date{2020/10/06}
%ここまで今回の記事関係ない
\usepackage{tcolorbox}
\tcbuselibrary{breakable, skins, theorems}

\theoremstyle{definition}
\newtheorem{thm}{定理}
\newtheorem{lem}[thm]{補題}
\newtheorem{prop}[thm]{命題}
\newtheorem{cor}[thm]{系}
\newtheorem{claim}[thm]{主張}
\newtheorem{dfn}[thm]{定義}
\newtheorem{rem}[thm]{注意}
\newtheorem{exa}[thm]{例}
\newtheorem{conj}[thm]{予想}
\newtheorem{prob}[thm]{問題}
\newtheorem{rema}[thm]{補足}

\DeclareMathOperator{\Ric}{Ric}
\DeclareMathOperator{\Vol}{Vol}
 \newcommand{\pdrv}[2]{\frac{\partial #1}{\partial #2}}
 \newcommand{\drv}[2]{\frac{d #1}{d#2}}
  \newcommand{\ppdrv}[3]{\frac{\partial #1}{\partial #2 \partial #3}}


%ここから本文.
\begin{document}
%\maketitle
\begin{center}
{\Large 第11回. 多重積分の変数変換公式 (川平先生の本, 第27章の内容)}
\end{center}

\begin{flushright}
 岩井雅崇, 2020/12/22
\end{flushright}


\section{変数変換公式}
 \begin{tcolorbox}[
    colback = white,
    colframe = green!35!black,
    fonttitle = \bfseries,
    breakable = true]
    \begin{dfn}
 $E \subset \R^2$を集合とする. 
 \underline{点$(a,b) \in \R^2$が$E$の境界}であるとは. 任意の正の数$r>0$について
 $B_{(a,b)}(r) = \{ (x,y)\in \R^2 :\sqrt{(x-a)^2 + (y-b)^2 } \leqq r  \}$とするとき, 
 $B_{(a,b)}(r) \cap E \neq \phi$かつ$B_{(a,b)}(r) \cap (\R^2 \setminus E) \neq \phi$となること.
\underline{ $E$の境界の点からなる集合を$\partial E$}とする.
 \end{dfn}
 \end{tcolorbox}
 \footnote{ $B_{(a,b)}(r) \cap E \neq \phi$ とは$B_{(a,b)}(r) \cap E$が空集合でないこと. つまり, ある元$(c,d) \in B_{(a,b)}(r) \cap E $が存在すること.}
 
 \begin{exa}
 $E = \{ (x,y)\in \R^2 :\sqrt{x^2 + y^2 } \leqq 1  \}$とする.このとき$E$の境界の点の集合は
 $$
 \partial E = \{ (x,y)\in \R^2 :\sqrt{x^2 + y^2 } = 1  \} \text{\,\,となる.}
 $$
 \end{exa}

 \begin{tcolorbox}[
    colback = white,
    colframe = green!35!black,
    fonttitle = \bfseries,
    breakable = true]
    \begin{dfn}
$E \subset \R^2$を面積確定な有界閉集合とし, 変数変換$\Phi$を次の通りとする.

 $$
\begin{array}{ccccc}
\Phi: &E & \rightarrow & \R^2 & \\
&(u,v) & \longmapsto & (x(u,v),y(u,v))&
\end{array}
$$

\underline{$\Phi$が重積分の変数変換の条件を満たす}とは, 次の条件(1)-(3)を満たすこと.

[条件(1).] $x(u,v),y(u,v)$は$C^1$級である.

[条件(2).] $D = \Phi(E)$とするとき, $E$の境界以外で$\Phi$は1対1写像. 

[条件(3).] $\Phi$のヤコビ行列
$$
D\Phi=
\left(\begin{array}{cc} \pdrv{x}{u} & \pdrv{x}{v} \\ \pdrv{y}{u}& \pdrv{y}{v} \\ \end{array} \right)
$$ 
とし, ヤコビアンを$\det D\Phi = 
\Bigl( \pdrv{x}{u} \Bigr) \left( \pdrv{y}{v} \right) - \Bigl( \pdrv{x}{v}  \Bigr) \left( \pdrv{y}{u}\right)$とするとき, $\det D\Phi $は$E$の境界以外で0にならない.
 \end{dfn}
 \end{tcolorbox}

\begin{exa}
\begin{itemize}
\item $E=[0,1]\times[0,1]$とし, 
 $$
\begin{array}{ccccc}
\Phi: &E & \rightarrow & \R^2 & \\
&(u,v) & \longmapsto & (u+v,v)&
\end{array}
$$
とすると, 条件(1)-(3)を満たす.
特に$
D\Phi=
\left(\begin{array}{cc} 1 &1 \\ 0& 1 \\ \end{array} \right)
$
かつ$\det D\Phi =1 \neq 0$である.

\item $E=[0,1]\times[0,1]$とし, 
 $$
\begin{array}{ccccc}
\Phi: &E & \rightarrow & \R^2 & \\
&(u,v) & \longmapsto & (u+v,u+v)&
\end{array}
$$
とすると, 条件(3)を満たさない.
実際$
D\Phi=
\left(\begin{array}{cc} 1 &1 \\ 1& 1 \\ \end{array} \right)
$
かつ$\det D\Phi =0$である.

\item $E=[0,1]\times[0,2\pi]$とし, 
 $$
\begin{array}{ccccc}
\Phi: &E & \rightarrow & \R^2 & \\
&(r,\theta) & \longmapsto & (r \cos \theta , r \sin \theta)&
\end{array}
$$
とすると, 条件(1)-(3)を満たし. $D = \Phi(E) =  \{ (x,y)\in \R^2 :\sqrt{x^2 + y^2 } \leqq 1  \}$となる.
特に$
D\Phi=
\left(\begin{array}{cc} \cos \theta  &-r \sin \theta \\ \sin \theta& r \cos \theta  \\ \end{array} \right)
$
かつ$\det D\Phi =r$である.\footnote{この例では$E \setminus \partial E = \{ (r,\theta)\in \R^2 : 0 < r<1, 0<\theta <2\pi\}$であるため, $E$の境界以外の集合である$E \setminus\partial E$上で$\det D\Phi =r$は0ではない.}

\item $E=[0,1]\times[0,4\pi]$とし, 
 $$
\begin{array}{ccccc}
\Phi: &E & \rightarrow & \R^2 & \\
&(r,\theta) & \longmapsto & (r \cos \theta , r \sin \theta)&
\end{array}
$$
とすると, 条件(2)を満たさない.
\footnote{$\Phi(\frac{1}{2}, \pi)=\Phi( \frac{1}{2}, 3\pi) =(-\frac{1}{2}, 0)$であるため1対1ではない. 1対1に関しては第4回授業を参照のこと.}
\end{itemize}

\end{exa}

 \begin{tcolorbox}[
    colback = white,
    colframe = green!35!black,
    fonttitle = \bfseries,
    breakable = true]
    \begin{thm}[多重積分の変数変換公式]
$E \subset \R^2$を面積確定な有界閉集合とし, 変数変換$\Phi$を次の通りとする.

 $$
\begin{array}{ccccc}
\Phi: &E & \rightarrow & \R^2 & \\
&(u,v) & \longmapsto & (x(u,v),y(u,v))&
\end{array}
$$
$\Phi$は重積分の変数変換の条件(条件(1)-(3))を満たすとする.

関数$f(x,y)$が$D = \Phi(E)$上で積分可能であるとき
$$
\iint_{D}f(x,y)dxdy=\iint_{E} f(x(u,v),y(u,v))|\det D \Phi|dudv \text{\,\,となる.}
$$
 \end{thm}
 \end{tcolorbox}

\begin{exa}
$D= \{ (x,y)\in \R^2 :\sqrt{x^2 + y^2 } \leqq 1  \}$とする. 
$\iint_{D} e^{-x^2-y^2}dxdy$を求めよ.

\hspace{-11pt}(解.) 
$E=[0,1]\times[0,2\pi]$とし, 
 $$
\begin{array}{ccccc}
\Phi: &E & \rightarrow & \R^2 & \\
&(r,\theta) & \longmapsto & (r \cos \theta , r \sin \theta)&
\end{array}
$$
とすると, 条件(1)-(3)を満たし. $D = \Phi(E)$かつ$\det D\Phi =r$である.

以上より多重積分の変数変換の公式から
\begin{align*}
\begin{split}
\iint_{D} e^{-x^2-y^2}dxdy
&=
\iint_{E} e^{-(r \cos \theta )^2- (r \sin \theta )^2} |r|drd\theta \\
&=
\iint_{E} e^{-r^2}r drd\theta \\
&=
\int_{0}^{2\pi} \left( \int_{0}^{1}e^{-r^2}r dr\right)d\theta 
=\int_{0}^{2\pi} \left[ \frac{-e^{-r^2}}{2} \right]_{0}^{1} d\theta 
=\int_{0}^{2\pi} \frac{1-e^{-1}}{2} d\theta =\pi\left( 1-\frac{1}{e}\right).
    \end{split}
  \end{align*}
\end{exa}

\begin{exa}
$D= \{ (x,y)\in \R^2 : |x+2y|\leqq1, |x-y|\leqq1 \}$とする. 
$\iint_{D} (x-y)^2dxdy$を求めよ.

\hspace{-11pt}(解.) 
$E=[-1,1]\times[-1,1]$とし, 
 $$
\begin{array}{ccccc}
\Phi: &E & \rightarrow & \R^2 & \\
&(u,v) & \longmapsto & (\frac{u+2v}{3} , \frac{u-v}{3} )&
\end{array}
$$
とすると, 条件(1)-(3)を満たし, 
$D = \Phi(E)$かつ$
D\Phi=
\left(\begin{array}{cc} \frac{1}{3} &\frac{2}{3} \\ \frac{1}{3}&-\frac{1}{3} \\ \end{array} \right)
$
かつ$\det D\Phi = -\frac{1}{3}\neq 0$である.
以上より多重積分の変数変換の公式から, 
\begin{align*}
\begin{split}
\iint_{D} (x-y)^2dxdy
&=
\iint_{E} \left(  \frac{u+2v}{3} - \frac{(u-v)}{3} \right)^2 \left| -\frac{1}{3}\right| dudv \\
&= 
\iint_{E} \frac{v^2}{3} dudv
=
\int_{-1}^{1} \left( \int_{-1}^{1} \frac{v^2}{3} dv\right)du 
=
\int_{-1}^{1} \left[ \frac{v^3}{9}\right]_{-1}^{1} du
=\int_{-1}^{1}  \frac{2}{9} du
=\frac{4}{9}. 
 \end{split}
  \end{align*}




\end{exa}
 
 

\end{document}
