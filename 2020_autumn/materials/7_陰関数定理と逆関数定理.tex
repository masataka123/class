\documentclass[dvipdfmx,a4paper,11pt]{article}
\usepackage[utf8]{inputenc}
%\usepackage[dvipdfmx]{hyperref} %リンクを有効にする
\usepackage{url} %同上
\usepackage{amsmath,amssymb} %もちろん
\usepackage{amsfonts,amsthm,mathtools} %もちろん
\usepackage{braket,physics} %あると便利なやつ
\usepackage{bm} %ラプラシアンで使った
\usepackage[top=30truemm,bottom=30truemm,left=25truemm,right=25truemm]{geometry} %余白設定
\usepackage{latexsym} %ごくたまに必要になる
\renewcommand{\kanjifamilydefault}{\gtdefault}
\usepackage{otf} %宗教上の理由でmin10が嫌いなので


\usepackage[all]{xy}
\usepackage{amsthm,amsmath,amssymb,comment}
\usepackage{amsmath}    % \UTF{00E6}\UTF{0095}°\UTF{00E5}\UTF{00AD}\UTF{00A6}\UTF{00E7}\UTF{0094}¨
\usepackage{amssymb}  
\usepackage{color}
\usepackage{amscd}
\usepackage{amsthm}  
\usepackage{wrapfig}
\usepackage{comment}	
\usepackage{graphicx}
\usepackage{setspace}
\setstretch{1.2}


\newcommand{\R}{\mathbb{R}}
\newcommand{\Z}{\mathbb{Z}}
\newcommand{\N}{\mathbb{N}}
\newcommand{\C}{\mathbb{C}} 



   %当然のようにやる.
\allowdisplaybreaks[4]
   %もちろん.
%\title{第1回. 多変数の連続写像 (岩井雅崇, 2020/10/06)}
%\author{岩井雅崇}
%\date{2020/10/06}
%ここまで今回の記事関係ない
\usepackage{tcolorbox}
\tcbuselibrary{breakable, skins, theorems}

\theoremstyle{definition}
\newtheorem{thm}{定理}
\newtheorem{lem}[thm]{補題}
\newtheorem{prop}[thm]{命題}
\newtheorem{cor}[thm]{系}
\newtheorem{claim}[thm]{主張}
\newtheorem{dfn}[thm]{定義}
\newtheorem{rem}[thm]{注意}
\newtheorem{exa}[thm]{例}
\newtheorem{conj}[thm]{予想}
\newtheorem{prob}[thm]{問題}
\newtheorem{rema}[thm]{補足}

\DeclareMathOperator{\Ric}{Ric}
\DeclareMathOperator{\Vol}{Vol}
 \newcommand{\pdrv}[2]{\frac{\partial #1}{\partial #2}}
 \newcommand{\drv}[2]{\frac{d #1}{d#2}}
  \newcommand{\ppdrv}[3]{\frac{\partial #1}{\partial #2 \partial #3}}


%ここから本文.
\begin{document}
%\maketitle
\begin{center}
{\Large 第7回. 陰関数定理と逆関数定理 (川平先生の本, 第24章の内容)}
\end{center}

\begin{flushright}
 岩井雅崇, 2020/11/24
\end{flushright}

\section{陰関数定理}
\begin{tcolorbox}[
    colback = white,
    colframe = green!35!black,
    fonttitle = \bfseries,
    breakable = true]
    \begin{thm}
    $f(x,y)$を$C^1$級関数とし, 点$(a,b)$で
    $f(a,b) =0 $かつ$\pdrv{f}{y}(a,b) \neq 0$とする.
    
この時$a$を含む開区間$I$と$I$上の$C^1$級関数$\phi : I \rightarrow \R$があって次の3つを満たす.
\begin{enumerate}
\item $b = \phi (a)$.
\item 任意の$x \in I$について, $f(x, \phi(x))=0$.
\item $\frac{d\phi}{dx} = \frac{-\pdrv{f}{x}(x,\phi(x)) }{\pdrv{f}{y}(x,\phi(x)) }$. 特に$\frac{d\phi}{dx}(a) = \frac{-\pdrv{f}{x}(a,b) }{\pdrv{f}{y}(a,b) }$.

$f(x,\phi(x))=0$となる関数$y=\phi(x)$を\underline{$f(x,y)=0$の陰関数}という.
\end{enumerate}
    \end{thm}
    \end{tcolorbox}

この定理によって, 陰関数が分からなくとも$\frac{d\phi}{dx}(a)$が計算できる.
\begin{exa}
$f(x,y) = x^3 -3xy+y^3-1$とする.
曲線$f(x,y)=0$の$(1,0)$での接線の方程式を求めよ.

(解.) 
$$
\pdrv{f}{x} = 3x^2-3y, \pdrv{f}{y} = -3x+3y^2 \text{\,\,である.}
$$
よって$\pdrv{f}{y}(1,0) \neq 0$より, 陰関数$\phi : I \rightarrow \R$があって, 
$$
\phi(1) =0, f(x,\phi(x)) =0, \frac{d\phi}{dx}(1) = \frac{-\pdrv{f}{x}(1,0) }{\pdrv{f}{y}(1,0) }=1.
$$
よって$y=\phi(x)$の$(1,0)$での接線の方程式は
$$
y = \frac{d\phi}{dx}(1)(x-1)=x-1 \text{\,\,\, である.}
$$
\end{exa}
   
 \section{逆関数定理}
 
 \begin{tcolorbox}[
    colback = white,
    colframe = green!35!black,
    fonttitle = \bfseries,
    breakable = true]
    \begin{thm}
    $\Phi$を領域$D$上の$C^1$級変数変換とし$D\Phi$を$\Phi$のヤコビ行列とする.
    
    $(a,b) \in D$で$\det(D\Phi(a,b)) \neq 0$ならば, 
    $(a,b)$を含む小さな円板上で$\Phi$は逆変換$\Phi^{-1}$をもち
    $D\Phi^{-1} = (D\Phi)^{-1}$となる.
    


    \end{thm}
    \end{tcolorbox}
    
逆関数定理から陰関数定理が導かれる.    
 

\end{document}
