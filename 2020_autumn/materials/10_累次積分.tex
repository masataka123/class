\documentclass[dvipdfmx,a4paper,11pt]{article}
\usepackage[utf8]{inputenc}
%\usepackage[dvipdfmx]{hyperref} %リンクを有効にする
\usepackage{url} %同上
\usepackage{amsmath,amssymb} %もちろん
\usepackage{amsfonts,amsthm,mathtools} %もちろん
\usepackage{braket,physics} %あると便利なやつ
\usepackage{bm} %ラプラシアンで使った
\usepackage[top=30truemm,bottom=30truemm,left=25truemm,right=25truemm]{geometry} %余白設定
\usepackage{latexsym} %ごくたまに必要になる
\renewcommand{\kanjifamilydefault}{\gtdefault}
\usepackage{otf} %宗教上の理由でmin10が嫌いなので


\usepackage[all]{xy}
\usepackage{amsthm,amsmath,amssymb,comment}
\usepackage{amsmath}    % \UTF{00E6}\UTF{0095}°\UTF{00E5}\UTF{00AD}\UTF{00A6}\UTF{00E7}\UTF{0094}¨
\usepackage{amssymb}  
\usepackage{color}
\usepackage{amscd}
\usepackage{amsthm}  
\usepackage{wrapfig}
\usepackage{comment}	
\usepackage{graphicx}
\usepackage{setspace}
\setstretch{1.2}


\newcommand{\R}{\mathbb{R}}
\newcommand{\Z}{\mathbb{Z}}
\newcommand{\N}{\mathbb{N}}
\newcommand{\C}{\mathbb{C}} 
\newcommand{\Area}{\text{Area}}





   %当然のようにやる.
\allowdisplaybreaks[4]
   %もちろん.
%\title{第1回. 多変数の連続写像 (岩井雅崇, 2020/10/06)}
%\author{岩井雅崇}
%\date{2020/10/06}
%ここまで今回の記事関係ない
\usepackage{tcolorbox}
\tcbuselibrary{breakable, skins, theorems}

\theoremstyle{definition}
\newtheorem{thm}{定理}
\newtheorem{lem}[thm]{補題}
\newtheorem{prop}[thm]{命題}
\newtheorem{cor}[thm]{系}
\newtheorem{claim}[thm]{主張}
\newtheorem{dfn}[thm]{定義}
\newtheorem{rem}[thm]{注意}
\newtheorem{exa}[thm]{例}
\newtheorem{conj}[thm]{予想}
\newtheorem{prob}[thm]{問題}
\newtheorem{rema}[thm]{補足}

\DeclareMathOperator{\Ric}{Ric}
\DeclareMathOperator{\Vol}{Vol}
 \newcommand{\pdrv}[2]{\frac{\partial #1}{\partial #2}}
 \newcommand{\drv}[2]{\frac{d #1}{d#2}}
  \newcommand{\ppdrv}[3]{\frac{\partial #1}{\partial #2 \partial #3}}


%ここから本文.
\begin{document}
%\maketitle
\begin{center}
{\Large 第10回. 累次積分 (川平先生の本, 第26章の内容)}
\end{center}

\begin{flushright}
 岩井雅崇, 2020/12/15
\end{flushright}
 
 \section{縦線領域と累次積分}
 
 \begin{tcolorbox}[
    colback = white,
    colframe = green!35!black,
    fonttitle = \bfseries,
    breakable = true]
    \begin{dfn}
   $\phi_1(x), \phi_{2}(x)$を$\phi_1(x) \leqq \phi_2(x)$となる$[a,b]$上の連続関数とする. 
   $$
D = \{ (x,y) \in \R^2 : a \leqq x \leqq b, \phi_1(x) \leqq y \leqq \phi_2(x)\}
$$
で表せられる領域を\underline{縦線領域}という.
 \end{dfn}
 \end{tcolorbox}
 
 
 
 \begin{exa}

\begin{itemize}
\item $D=[a,b]\times [c,d]$は縦線領域. $\phi_1(x)=c$, $\phi_2(x)=d$とすれば良い.
\item 縦線領域
$D = \{ (x,y) \in \R^2 :\ -1 \leqq x \leqq 1, -\sqrt{1-x^2} \leqq y \leqq \sqrt{1-x^2} \}$
とおくと$D$は原点中心の半径1の円.
\end{itemize}
\end{exa}
 
 \begin{tcolorbox}[
    colback = white,
    colframe = green!35!black,
    fonttitle = \bfseries,
    breakable = true]
    \begin{thm}
縦線領域を$D = \{ (x,y) \in \R^2 : a \leqq x \leqq b, \phi_1(x) \leqq y \leqq \phi_2(x)\}$とし, $f(x,y)$
を$D$上の連続関数とする.
このとき, $f(x,y)$は$D$上で積分可能であり,
$$
\iint_{D} f(x,y)dxdy = \int_{a}^{b }\left( \int_{\phi_1(x)}^{\phi_{2}(x)}f(x,y)dy    \right) dx \text{\,\,となる.}
$$
これを\underline{$f(x,y)$の累次積分}という.

特に$D$は面積確定で
$$
\Area(D) = \iint_{D} dxdy=\int_{a}^{b} \{ \phi_2(x) - \phi_1(x)\}dx.
$$
        \end{thm}
 \end{tcolorbox}
 
  \begin{exa}
 $D=[0,1]\times[0,1]$, $f(x,y)=x^2+y^2$とする.
 $\iint_{D}f(x,y)dxdy$を求めよ.

\hspace{-11pt}(解.) $\phi_1(x)=0, \phi_2(x)=1$とすると上の定理より, 

$$
%  \iint_{D}f(x,y)dxdy %&=  \int_{0}^{1}\left( \int_{0}^{1} x^2+y^2dy    \right) dx \\ &
\iint_{D}x^2+y^2dxdy 
  =  \int_{0}^{1}\left( \int_{0}^{1} x^2+y^2dy    \right) dx 
  = \int_{0}^{1}\left[ x^{2}y + \frac{y^3}{3}   \right]_{0}^{1} dx 
  =\int_{0}^{1} x^{2}+ \frac{1}{3}  dx 
  =\left[  \frac{x^3}{3} +\frac{x}{3}  \right]_{0}^{1} =\frac{2}{3}.
$$
 \end{exa}
 
   \begin{exa}
$D = \{ (x,y) \in \R^2 : -1 \leqq x \leqq 1, -\sqrt{1-x^2} \leqq y \leqq \sqrt{1-x^2} \}$とする.
$\Area(D) = \iint_{D} dxdy$を求めよ.

\hspace{-11pt}(解.) $\phi_1(x)=-\sqrt{1-x^2}, \phi_2(x)=\sqrt{1-x^2} $とすると上の定理より, 

$$
  \iint_{D}dxdy =\int_{-1}^{1} \left\{ \sqrt{1-x^2} - \left(-\sqrt{1-x^2} \right) \right\} dx
  = 2 \int_{-1}^{1} \sqrt{1-x^2} dx =\pi.
$$
  \end{exa}
つまり半径1の円の面積は$\pi$.
 
    \begin{exa}
$D = \{ (x,y) \in \R^2 : 0 \leqq x \leqq 1, x^2 \leqq y \leqq 1 \}$とし, 
$f(x,y)=xe^{-y^2}$とするとき,  $\iint_{D}f(x,y)dxdy$を求めよ.

\hspace{-11pt}(解.) 普通に定理を適用すると, 
$$
\iint_{D} xe^{-y^2}dxdy = \int_{0}^{1} \left( \int_{x^2}^{1} xe^{-y^2}dy    \right) dx
$$
となるが, $e^{-y^2}$の不定積分がわからないため, ここで手詰まりとなる.

そこで$D = \{ (x,y) \in \R^2 : 0 \leqq y \leqq 1, 0 \leqq x \leqq \sqrt{y} \}$に注意すると, 
\begin{align*}
\begin{split}
\iint_{D} xe^{-y^2}dxdy &= \int_{0}^{1} \left( \int_{0}^{\sqrt{y}} xe^{-y^2}dx   \right) dy \\
&= \int_{0}^{1}\left[ \frac{x^{2} e^{-y^2} }{2}  \right]_{0}^{\sqrt{y}} dy
=  \int_{0}^{1}    \frac{y e^{-y^2} }{2}      dy
= \left[  \frac{- e^{-y^2} }{4}   \right]_{0}^{1} = \frac{1}{4}\left( 1 - \frac{1}{e}\right).
\end{split}
\end{align*}

  \end{exa}

\end{document}
