\documentclass[dvipdfmx,a4paper,11pt]{article}
\usepackage[utf8]{inputenc}
%\usepackage[dvipdfmx]{hyperref} %リンクを有効にする
\usepackage{url} %同上
\usepackage{amsmath,amssymb} %もちろん
\usepackage{amsfonts,amsthm,mathtools} %もちろん
\usepackage{braket,physics} %あると便利なやつ
\usepackage{bm} %ラプラシアンで使った
\usepackage[top=30truemm,bottom=30truemm,left=25truemm,right=25truemm]{geometry} %余白設定
\usepackage{latexsym} %ごくたまに必要になる
\renewcommand{\kanjifamilydefault}{\gtdefault}
\usepackage{otf} %宗教上の理由でmin10が嫌いなので


\usepackage[all]{xy}
\usepackage{amsthm,amsmath,amssymb,comment}
\usepackage{amsmath}    % \UTF{00E6}\UTF{0095}°\UTF{00E5}\UTF{00AD}\UTF{00A6}\UTF{00E7}\UTF{0094}¨
\usepackage{amssymb}  
\usepackage{color}
\usepackage{amscd}
\usepackage{amsthm}  
\usepackage{wrapfig}
\usepackage{comment}	
\usepackage{graphicx}
\usepackage{setspace}
\setstretch{1.2}


\newcommand{\R}{\mathbb{R}}
\newcommand{\Z}{\mathbb{Z}}
\newcommand{\N}{\mathbb{N}}
\newcommand{\C}{\mathbb{C}} 



   %当然のようにやる.
\allowdisplaybreaks[4]
   %もちろん.
%\title{第1回. 多変数の連続写像 (岩井雅崇, 2020/10/06)}
%\author{岩井雅崇}
%\date{2020/10/06}
%ここまで今回の記事関係ない
\usepackage{tcolorbox}
\tcbuselibrary{breakable, skins, theorems}

\theoremstyle{definition}
\newtheorem{thm}{定理}
\newtheorem{lem}[thm]{補題}
\newtheorem{prop}[thm]{命題}
\newtheorem{cor}[thm]{系}
\newtheorem{claim}[thm]{主張}
\newtheorem{dfn}[thm]{定義}
\newtheorem{rem}[thm]{注意}
\newtheorem{exa}[thm]{例}
\newtheorem{conj}[thm]{予想}
\newtheorem{prob}[thm]{問題}
\newtheorem{rema}[thm]{補足}

\DeclareMathOperator{\Ric}{Ric}
\DeclareMathOperator{\Vol}{Vol}
 \newcommand{\pdrv}[2]{\frac{\partial #1}{\partial #2}}
 \newcommand{\drv}[2]{\frac{d #1}{d#2}}
  \newcommand{\ppdrv}[3]{\frac{\partial #1}{\partial #2 \partial #3}}


%ここから本文.
\begin{document}
%\maketitle
\begin{center}
{ \large 大阪市立大学 R2年度(2020年度)後期  全学共通科目 解析II TI機・情33 $\sim$} \\
\vspace{5pt}

{\LARGE 中間レポート } \\
\vspace{5pt}

{ \Large 提出締め切り 2020年12月22日 23時59分00秒 (日本標準時刻)}
\end{center}

\begin{flushright}
 担当教官: 岩井雅崇(いわいまさたか) 
\end{flushright}

{\Large $\bullet$ 注意事項}
\begin{enumerate}
\item 第1問から第4問まで解くこと. 
\item おまけ問題は全員が解く必要はない.(詳しくは成績の付け方のスライドを参照せよ).
\item 用語に関しては授業または教科書(川平友規著 微分積分 1変数と2変数)に準じます.
\item \underline{提出締め切りを遅れて提出した場合, 大幅に減点する可能性がある.}
\item \underline{名前・学籍番号をきちんと書くこと.}
\item \underline{解答に関して, 答えのみならず, 答えを導出する過程をきちんと記してください.} きちんと記していない場合は大幅に減点する場合がある.
%ただし用語の定義の違いによるミスに関して, 大幅に減点することはない.
\item 字は汚くても構いませんが, \underline{読める字で濃く書いてください.} あまりにも読めない場合は採点をしないかもしれません.%\footnote{私も字が汚い方ですので人のこと言えませんが...自論ですが, 字が汚いと自覚ある人は大きく書けば読みやすくなると思います.}
\item 採点を効率的に行うため, \underline{順番通り解答するようお願いいたします.}
\item 採点を効率的に行うため,  \underline{レポートはpdfファイル形式で提出し,} ファイル名を「dif(学籍番号).pdf」とするようお願いいたします. 
(difは微分(differential)の略です.)
例えば学籍番号が「A18CA999」の場合はファイル名は「difA18CA999.pdf」となります.
\end{enumerate}

 \begin{tcolorbox}[
    colback = white,
    colframe = black,
    fonttitle = \bfseries,
    breakable = true]
    レポート提出前のチェックリスト
    \begin{itemize}
    \item[] $\Box$ 締め切りを守っているか?
    \item[] $\Box$ レポートに名前・学籍番号を書いたか?
     \item[] $\Box$ 答えを導出する過程をきちんと記したか?
    \item[] $\Box$ 他者が読める字で書いたか?
    \item[] $\Box$ 順番通り解答したか?
    \item[] $\Box$レポートはpdfファイル形式で提出したか?
   \item[] $\Box$ ファイル名を「dif(学籍番号).pdf」としたか?
    \end{itemize}

  \end{tcolorbox}
  
2020年12月15日(火)の10時50分からオンラインによる質疑応答の場を設けます. (出席義務はありません, 来たい人だけ来てください. レポートに関する質問も可とします.) 質疑応答に関してはWebClassを参照してください.
 
\newpage
 \hspace{-11pt}
{\Large $\bullet$ レポートの提出方法について }
\vspace{11pt}

\underline{原則的にWebClassからの提出しか認めません.}
レポートは余裕を持って提出してください.
\vspace{11pt}

\underline{レポートはpdfファイルで提出してください.}
またWebClassからの提出の際, 提出ファイルを一つにまとめる必要があるとのことですので, 提出ファイルを一つにまとめてください.
\vspace{11pt}

\underline{採点を効率的に行うため, ファイル名を「dif(学籍番号).pdf」とするようお願いいたします.}
(difは微分(differential)の略です.)
例えば学籍番号が「A18CA999」の場合はファイル名は「difA18CA999.pdf」となります.

\vspace{11pt}
 \hspace{-11pt}
{\Large $\bullet$ 提出用pdfファイルの作成の仕方について}
\vspace{11pt}

いろいろ方法はあると思います.
\vspace{11pt}

1つ目は「手書きレポートをpdfにする方法」があります.
この方法は時間はあまりかかりませんが, お金がかかる可能性があります.
手書きレポートをpdfにするには以下の方法があると思います.
\begin{itemize}
\setlength{\parskip}{0cm} % 段落間
  \setlength{\itemsep}{0cm}
\item スキャナーを使うかコンビニに行ってスキャンする.
\item スマートフォンやカメラで画像データにしてからpdfにする. 例えばMicrosoft Wordを使えば画像データをpdfにできます. %(見づらくなる可能性あり)
\item その他いろいろ検索して独自の方法を行う.
\end{itemize}

2つ目は「TeXでレポートを作成する方法」があります.
時間はかなりかかりますが, 見た目はかなり綺麗です.
\vspace{11pt}


いずれの方法でも構いません. 最終的に私が読めるように書いたレポートであれば大丈夫です.
%他者が読める字で書いてあれば問題ありません. (私が読めるようなレポートであれば大丈夫です.)

\vspace{11pt}
 \hspace{-11pt}
{\Large $\bullet$ WebClassからの提出が不可能な場合}
\vspace{11pt}

提出の期限までに (WebClassのシステムトラブル等の理由で) WebClassからの提出が不可能な場合のみメール提出を受け付けます.
その場合には以下の項目を厳守してください.
\begin{itemize}
\setlength{\parskip}{0cm} % 段落間
  \setlength{\itemsep}{0cm}
\item 大学のメールアドレスを使って送信すること. (なりすまし提出防止のため.)
\item 件名を「レポート提出」とすること
\item 講義名, 学籍番号, 氏名 (フルネーム)を書くこと.
\item レポートのファイルを添付すること.
\item WebClassでの提出ができなかった事情を説明すること. (提出理由が不十分である場合, 減点となる可能性があります.)
\end{itemize}

メール提出の場合はmasataka[at]sci.osaka-cu.ac.jpにメールするようお願いいたします.

%正当な理由(WebClassのシステムトラブル等)ではない場合, メールでの提出は減点対象となるので注意すること.
\newpage
\begin{center}
{\LARGE 中間レポート問題.} 
\end{center}

{\Large 第1問.} (授業第2回の内容.)
\vspace{11pt}

$\R^2$上の関数$f(x,y)$を以下で定める.
$$
  f(x,y)= \begin{cases}
     \frac{(x-y)^3}{x^2+y^2}& (x,y) \neq (0,0) \text{\,\,のとき.} \\
    0&  (x,y) = (0,0) \text{\,\,のとき.} 
  \end{cases}
  $$
  
(1).
 $f(x,y)$が点$(0,0)$で偏微分可能であることを示せ.
 
(2).
  $f(x,y)$が点$(0,0)$で全微分可能ではないことを示せ.
  
 \vspace{11pt}
 
(3).
$\R^2$上の$C^1$級関数を$g(x,y) = x^3+2xy^2+y-11$とおく. \\
 \hspace{33pt}$g(x,y)$の点$(1,2)$での接平面の方程式を求めよ.

 \vspace{33pt}
 
 {\Large 第2問.} (授業第3回の内容.)
 \vspace{11pt}
 
 $\R$上の$C^1$級関数$\text{cosh}(t)$と$\text{sinh}(t)$を
 $$
 \text{cosh}(t) = \frac{e^t+e^{-t}}{2} \text{,\,\,\,\,\,} \text{sinh}(t) = \frac{e^t - e^{-t}}{2} \text{\,\, とする.}
 $$
  $f(x,y)$を $\R^2$上の$C^1$級関数とし, $C^1$級変数変換を$(x(r,t),y(r,t)) = (r \text{ cosh}(t), r\text{ sinh}(t))$とする. \\
  $g(r,t) = f(x(r,t), y(r,t))$とするとき, $\pdrv{g}{r}, \pdrv{g}{t}$を$\pdrv{f}{x},\pdrv{f}{y}$を用いてあらわせ.
  
   \vspace{33pt}
   
   {\Large 第3問.} (授業第6回の内容.)
    \vspace{11pt}
    
$\R^2$上の$C^2$級関数を$f(x,y) = x^3 + y^3 + 6xy$とする.
$f(x,y) $について極大点・極小点を持つ点があれば, その座標と極値を求めよ. またその極値が極大値か極小値のどちらであるか示せ.
  
     \vspace{33pt} 
   
{\Large 第4問.} (授業第8回の内容.)
    \vspace{11pt}

$\R^3$上の$C^{\infty}$級関数を$f(x,y,z) = x+y+z$, $g(x,y,z) = x^2+2y^2+z^{2}-1$とする. \\ 
$g(x,y,z)=0$のもとで, $f(x,y,z)$の最大値と最大値をとる点の座標, 最小値と最小値をとる点の座標を全て求めよ. \\

    \vspace{-11pt}
    
つまり$S = \{ (x,y,z) \in \R^3: g(x,y,z)=0\}$とするとき, 
$f(x,y,z)$の$S$上での最大値と最大値をとる点の座標, 最小値と最小値をとる点の座標を全て求めよ. \\

   \vspace{-11pt}
 ただし, $S$上で$f(x,y,z)$が最大値・最小値をとることは認めて良い.
 
 \newpage 
 
{\Large 中間レポートおまけ問題.} (授業第1回の内容.)
\vspace{11pt}

$\R^2$上の部分集合$E$を次で定める.

$$
E = \{ (x,y) \in \R^2 : \text{$x$と$y$は共に有理数.}\}
$$

(1).
$E$が開集合ではないことを示せ.
 
(2).
$E$が閉集合ではないことを示せ.
    \vspace{11pt}


ただし次の事実は認めて良い.

(事実.[有理数の稠密性])
$a<b$なる任意の実数$a,b$について, $a<q<b$となる有理数$q$が存在する.
%任意の点$(a,b)\in \R^2$と任意の正の数$r>0$について, 点$(a,b)$中心の半径$r$の円板を$$B =\{ (x,y) \in \R^2 : \sqrt{(x-a)^2 + (y-b)^2} \leqq r\} \text{\,\, とするとき, }$$  ある点$(c,d) \in B$があって, $c$も$d$も共に有理数である.

     \vspace{33pt} 
     
 \begin{flushright}
 {\LARGE 以上.}
 \end{flushright}


%$f(x,y)$を$C^1$級関数とし,$C^1$級変数変換を$(x(u,v),y(u,v)) = (u \cos v, u \sin v)$とする.$g(u,v) = f(x(u,v), y(u,v))$とする時, $\pdrv{g}{u}, \pdrv{g}{v}$を$\pdrv{f}{x},\pdrv{f}{y}$を用いてあらわせ.

%$f(x,y) = x^3 -y^3 -3x +12y$について極大点・極小点を持つ点があれば, その座標と極値を求めよ. またその極値が極小値か極大値のどちらであるか示せ.

 %$f(x,y,z)= xyz, g(x,y,z)=x+y+z-170$とする.$g(x,y,z)=0$のもとで$f$の最大値を求めよ.つまり$S : = \{ (x,y) \in \R^3: g(x,y,z)=0\}$とする時, $f$の$S$上での最大値を求めよ.ただし$f$が$S$上で最大値を持つことは認めて良い.
  
 %

 

\end{document}
