\documentclass[dvipdfmx,a4paper,11pt]{article}
\usepackage[utf8]{inputenc}
%\usepackage[dvipdfmx]{hyperref} %リンクを有効にする
\usepackage{url} %同上
\usepackage{amsmath,amssymb} %もちろん
\usepackage{amsfonts,amsthm,mathtools} %もちろん
\usepackage{braket,physics} %あると便利なやつ
\usepackage{bm} %ラプラシアンで使った
\usepackage[top=30truemm,bottom=30truemm,left=25truemm,right=25truemm]{geometry} %余白設定
\usepackage{latexsym} %ごくたまに必要になる
\renewcommand{\kanjifamilydefault}{\gtdefault}
\usepackage{otf} %宗教上の理由でmin10が嫌いなので



\usepackage[all]{xy}
\usepackage{amsthm,amsmath,amssymb,comment}
\usepackage{amsmath}    % \UTF{00E6}\UTF{0095}°\UTF{00E5}\UTF{00AD}\UTF{00A6}\UTF{00E7}\UTF{0094}¨
\usepackage{amssymb}  
\usepackage{color}
\usepackage{amscd}
\usepackage{amsthm}  
\usepackage{wrapfig}
\usepackage{comment}	
\usepackage{graphicx}
\usepackage{setspace}
\setstretch{1.2}


\newcommand{\R}{\mathbb{R}}
\newcommand{\Z}{\mathbb{Z}}
\newcommand{\N}{\mathbb{N}}
\newcommand{\C}{\mathbb{C}} 



   %当然のようにやる.
\allowdisplaybreaks[4]
   %もちろん.
%\title{第1回. 多変数の連続写像 (岩井雅崇, 2020/10/06)}
%\author{岩井雅崇}
%\date{2020/10/06}
%ここまで今回の記事関係ない
\usepackage{tcolorbox}
\tcbuselibrary{breakable, skins, theorems}

\theoremstyle{definition}
\newtheorem{thm}{定理}
\newtheorem{lem}[thm]{補題}
\newtheorem{prop}[thm]{命題}
\newtheorem{cor}[thm]{系}
\newtheorem{claim}[thm]{主張}
\newtheorem{dfn}[thm]{定義}
\newtheorem{rem}[thm]{注意}
\newtheorem{exa}[thm]{例}
\newtheorem{conj}[thm]{予想}
\newtheorem{prob}[thm]{問題}
\newtheorem{rema}[thm]{補足}

\DeclareMathOperator{\Ric}{Ric}
\DeclareMathOperator{\Vol}{Vol}
 \newcommand{\pdrv}[2]{\frac{\partial #1}{\partial #2}}



%ここから本文.
\begin{document}
%\maketitle
\begin{center}
{\Large 第1回. 多変数の連続写像 (川平先生の本, 第16章の内容)}
\end{center}

\begin{flushright}
 岩井雅崇, 2020/10/06
\end{flushright}



\section{いくつかの準備}
以下の用語に関して興味のない人は読み飛ばして良い.

\begin{enumerate}
\item $xy$平面上の点$(a,b)$と正の数$r>0$について, 点$(a,b)$中心の半径$r$の(閉)円板を
$$
B_{(a,b)}(r) = \{ (x,y) \in \R^2 : \sqrt{(x-a)^2 + (y-b)^2} \leqq r  \}
\text{\,\, とする.} 
$$

\item $a<b$かつ$c<d$について, 
$$
[a,b] \times [c,d] = \{ (x,y) \in \R^2 : a \leqq x \leqq b \text{かつ}  c \leqq y \leqq d \}
\text{\,\, とする.} 
$$
\end{enumerate}
以下$E \subset \R^2$を集合とする.

\begin{enumerate}
\setcounter{enumi}{2}
\item $(a,b) \in E$が$E$の\underline{内点}とは, ある正の数$r>0$があって$B_{(a,b)}(r) \subset E$となること.
%\item $(a,b) \in E$が$E$の\underline{境界点}とは, 任意の$r>0$について$B_{(a,b)}(r) \cap E \neq \phi$かつ$B_{(a,b)}(r) \cap (\R^2 \setminus E) \neq \phi$となること.
\item $E$が\underline{開集合}とは, 任意の(全ての)$(a,b) \in E$について$(a,b)$は$E$の内点となること.
\item $E$が\underline{閉集合}とは, $\R^2 \setminus E$\footnote{$\R^2 \setminus E = \{ (x,y) \in \R^2 : (x,y) \not \in E\} $と定義する.}
が開集合であること.
\item $E$が\underline{有界}とは, ある正の数$M>0$があって, $ E \subset [-M,M] \times [-M,M]$となること.
\item $E$が\underline{連結}とは, $E$の任意の2点が$E$内の折れ線で結べること.
\item $E$が\underline{領域}とは, $E$が連結な開集合であること.
\end{enumerate}

\begin{exa}
%\text{}

\begin{itemize}
\item $\{ (x,y) \in \R^2 : \sqrt{x^2 + y^2} < 1  \}$は開集合, 有界, 連結, 領域. でも閉集合ではない.
\item $\{ (x,y) \in \R^2 : \sqrt{x^2 + y^2} \leqq 1 \}$は閉集合, 有界, 連結. でも開集合ではない.
\item $\R^2$は開集合, 閉集合, 連結, 領域. でも有界ではない.
\end{itemize}
\end{exa}

\begin{enumerate}
\setcounter{enumi}{8}
\item 関数$f(x,y)$の値が定まる集合を$f$の\underline{定義域}といい, 
$$\{ k \in \R : f(x,y) = k \text{となる$(x,y)$が定義域内に存在}\}$$
を$f$の\underline{値域}という.
\item \underline{$f$が領域$E$上の関数}とは, $E$が$f$の定義域に含まれることである.
このとき
$$
\begin{array}{ccccc}
f: &E & \rightarrow & \R & \text{とかく.}\\
&(x,y) & \longmapsto & f(x,y)&
\end{array}
$$
\end{enumerate}

\begin{exa}
$f(x,y) = \sqrt{1-x^2-y^2}$の定義域は$\{ (x,y) \in \R^2 : \sqrt{x^2 + y^2} \leqq 1  \}$. 値域は$[0,1]$.

\end{exa}

\section{極限と連続性}


\begin{tcolorbox}[
    colback = white,
    colframe = green!35!black,
    fonttitle = \bfseries,
    breakable = true]
    \begin{dfn}
    \underline{$(x,y)$が$(a,b)$に限りなく近づく}とは$(x,y) \neq (a,b)$かつ
    $$ \sqrt{(x-a)^2 + (y-b)^2} \rightarrow 0$$
    となるように変化すること.
    以後, $(x,y) \rightarrow (a,b)$とかく.
    \end{dfn}
\end{tcolorbox}

\begin{tcolorbox}[
    colback = white,
    colframe = green!35!black,
    fonttitle = \bfseries,
    breakable = true]
    \begin{dfn}
    $f(x,y)$を領域$D$上の関数とする.
    \underline{$f(x,y)$が$(x,y) \rightarrow (a,b)$のとき実数$A$に収束する}とは
    $(x,y)$が$(a,b)$に近づくとき, $f(x,y)$が$A$に限りなく近づくことである.
    このとき
    $$ \lim_{(x,y) \rightarrow (a,b)} f(x,y) =A \text{\,\, または\,\,\,}
    f(x,y) \rightarrow A  \text{\,\,}((x,y) \rightarrow (a,b)) \text{\,\, とかく.\,\,}
    $$ 
   
  
    \end{dfn}
\end{tcolorbox}

\begin{tcolorbox}[
    colback = white,
    colframe = green!35!black,
    fonttitle = \bfseries,
    breakable = true]
    \begin{dfn}
    $f(x,y)$を領域$D$上の関数とする.
    \underline{$f$が$(a,b)\in D$で連続}とは
    $$ \lim_{(x,y) \rightarrow (a,b)} f(x,y) =f(a,b) \text{\,\, となること.\,\,}$$
    \underline{$f$が$D$上で連続}とは$f$が任意の点$(a,b)\in D$で連続となること.
    \end{dfn}
\end{tcolorbox}

\begin{exa}
\label{conti}
\begin{itemize}
\item $ f(x,y) = x+y$, $f(x,y) = e^{x+y^2}$. これらは$\R^2$上の連続関数. (みんながよく知っている関数は連続関数.)
\item $f(x,y) = \frac{1}{x^2 + y^2}$は定理\ref{continess}より$\R^2 \setminus \{ (0,0)\}$上の連続関数. 
\item $\R^2$上の関数$f(x,y)$を次で定義する.
$$
  f(x,y)= \begin{cases}
     \frac{x^2-y^2}{x^2+y^2}& ((x,y) \neq (0,0)) \\
    1& ((x,y) = (0,0)) 
  \end{cases}
  $$
% 定理\ref{continess}%より, $f$は$\R^2 \setminus \{ (0,0)\}$上の連続関数.
  
$f$は$(0,0)$で連続ではない.

なぜなら$(0,y) \rightarrow (0,0)$という近づけ方をすると, $f(0,y) = -1 \rightarrow -1 $
%\text{\,\,}((0,y) \rightarrow (0,0))$
であるため, \\
$ \lim_{(0,y) \rightarrow (0,0)} f(0,y) \neq f(0,0)$より連続ではない.

\end{itemize}
\end{exa}

\begin{tcolorbox}[
    colback = white,
    colframe = green!35!black,
    fonttitle = \bfseries,
    breakable = true]
    \begin{thm}
     $\lim_{(x,y) \rightarrow (a,b)} f(x,y) =A$, $ \lim_{(x,y) \rightarrow (a,b)} g(x,y) =B$とする. このとき以下が成り立つ.
     
 $$  \lim_{(x,y) \rightarrow (a,b)} \left\{ f(x,y) + g(x,y)\right\} =A+B.$$
  $$  \lim_{(x,y) \rightarrow (a,b)} f(x,y) g(x,y) =AB.$$.
   $$\text{$B \neq 0$のとき, }  \lim_{(x,y) \rightarrow (a,b)} \frac{f(x,y) }{ g(x,y)} =\frac{A}{B}.$$
    \end{thm}
\end{tcolorbox}
\begin{tcolorbox}[
    colback = white,
    colframe = green!35!black,
    fonttitle = \bfseries,
    breakable = true]
    \begin{thm}
    \label{continess}
    関数$f,g$が点$(a,b)$で連続であるとする. このとき以下が成り立つ.
    \begin{itemize}
    \item $f(x,y) + g(x,y)$や$f(x,y)g(x,y)$は$(a,b)$で連続.
    \item $g(a,b) \neq 0$のとき, $\frac{f(x,y) }{ g(x,y)} $は$(a,b)$で連続.
    \end{itemize}

    \end{thm}
\end{tcolorbox}



\section{最大最小の存在}
\begin{tcolorbox}[
    colback = white,
    colframe = green!35!black,
    fonttitle = \bfseries,
    breakable = true]
    \begin{thm}
    有界な閉集合$D$上で連続な関数$f$は最大値・最小値を持つ.
    \end{thm}
\end{tcolorbox}

\begin{exa}
$D = \{ (x,y) \in \R^2 : \sqrt{x^2 + y^2} \leqq 1  \}$とすると$D$は有界閉集合.
$f(x,y) =x$とすると$f$は連続. 
実際, $D$上で$f$は最大値 1, 最小値 -1を持つ.
\end{exa}


%$\pdrv{f}{x}$

\end{document}
