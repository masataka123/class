\documentclass[dvipdfmx,a4paper,11pt]{article}
\usepackage[utf8]{inputenc}
%\usepackage[dvipdfmx]{hyperref} %リンクを有効にする
\usepackage{url} %同上
\usepackage{amsmath,amssymb} %もちろん
\usepackage{amsfonts,amsthm,mathtools} %もちろん
\usepackage{braket,physics} %あると便利なやつ
\usepackage{bm} %ラプラシアンで使った
\usepackage[top=30truemm,bottom=30truemm,left=25truemm,right=25truemm]{geometry} %余白設定
\usepackage{latexsym} %ごくたまに必要になる
\renewcommand{\kanjifamilydefault}{\gtdefault}
\usepackage{otf} %宗教上の理由でmin10が嫌いなので


\usepackage[all]{xy}
\usepackage{amsthm,amsmath,amssymb,comment}
\usepackage{amsmath}    % \UTF{00E6}\UTF{0095}°\UTF{00E5}\UTF{00AD}\UTF{00A6}\UTF{00E7}\UTF{0094}¨
\usepackage{amssymb}  
\usepackage{color}
\usepackage{amscd}
\usepackage{amsthm}  
\usepackage{wrapfig}
\usepackage{comment}	
\usepackage{graphicx}
\usepackage{setspace}
\setstretch{1.2}


\newcommand{\R}{\mathbb{R}}
\newcommand{\Z}{\mathbb{Z}}
\newcommand{\N}{\mathbb{N}}
\newcommand{\C}{\mathbb{C}} 



   %当然のようにやる.
\allowdisplaybreaks[4]
   %もちろん.
%\title{第1回. 多変数の連続写像 (岩井雅崇, 2020/10/06)}
%\author{岩井雅崇}
%\date{2020/10/06}
%ここまで今回の記事関係ない
\usepackage{tcolorbox}
\tcbuselibrary{breakable, skins, theorems}

\theoremstyle{definition}
\newtheorem{thm}{定理}
\newtheorem{lem}[thm]{補題}
\newtheorem{prop}[thm]{命題}
\newtheorem{cor}[thm]{系}
\newtheorem{claim}[thm]{主張}
\newtheorem{dfn}[thm]{定義}
\newtheorem{rem}[thm]{注意}
\newtheorem{exa}[thm]{例}
\newtheorem{conj}[thm]{予想}
\newtheorem{prob}[thm]{問題}
\newtheorem{rema}[thm]{補足}

\DeclareMathOperator{\Ric}{Ric}
\DeclareMathOperator{\Vol}{Vol}
 \newcommand{\pdrv}[2]{\frac{\partial #1}{\partial #2}}
 \newcommand{\drv}[2]{\frac{d #1}{d#2}}
  \newcommand{\ppdrv}[3]{\frac{\partial #1}{\partial #2 \partial #3}}


%ここから本文.
\begin{document}
%\maketitle
\begin{center}
{\Large 第5回. テイラー展開 (川平先生の本, 第22章の内容)}
\end{center}

\begin{flushright}
 岩井雅崇, 2020/11/10
\end{flushright}


\begin{tcolorbox}[
    colback = white,
    colframe = green!35!black,
    fonttitle = \bfseries,
    breakable = true]
    \begin{thm}
    $f$を領域$D$上の$C^2$級関数とし, $(a,b)  \in D$とする.
    点$(a,b)$中心の半径$r>0$の円板$B \subset D$を一つとる.
    
    任意の$(x,y) \in B$について$(a,b)$と$(x,y) $を結ぶ線分上の点$(a',b')$があって,
  \begin{align*}
  \begin{split}
  f(x,y) &= f(a,b) + \pdrv{f}{x}(a,b)(x-a) + \pdrv{f}{y}(a,b)(y-b) \\
  &+ \frac{1}{2} \left\{  \pdrv{^2f}{x^2}(a',b')(x-a)^2 +2 \ppdrv{^2f}{x}{y}(a',b')(x-a)(y-b)+
   \pdrv{^2f}{y^2}(a',b')(y-b) ^2    \right\}.
    \end{split}
  \end{align*}

    \end{thm}
    \end{tcolorbox}
    
    
\begin{tcolorbox}[
    colback = white,
    colframe = green!35!black,
    fonttitle = \bfseries,
    breakable = true]
    \begin{thm}
    $f$を領域$D$上の$C^{\infty}$級関数とし, $(a,b)  \in D$とする.
    点$(a,b)$中心の半径$r>0$の円板$B \subset D$を一つとる.
    
    任意の$(x,y) \in B$について$(a,b)$と$(x,y) $を結ぶ線分上の点$(a',b')$があって,
  \begin{align*}
  \begin{split}
  f(x,y) &= f(a,b) + \pdrv{f}{x}(a,b)(x-a) + \pdrv{f}{y}(a,b)(y-b) \\
  &+ \frac{1}{2} \left\{  \pdrv{^2f}{x^2}(a,b)(x-a)^2 +2 \ppdrv{^2f}{x}{y}(a,b)(x-a)(y-b)+
   \pdrv{^2f}{y^2}(a,b)(y-b) ^2    \right\}\\
   &+ \cdots \\
   &+ \frac{1}{n!}\left\{ \sum_{i=0}^{n} {}_n C_r \ppdrv{^n f}{x^i }{ y^{n-i}} (a', b') (x-a)^{i}(y-b)^{n-i}\right\}.
    \end{split}
  \end{align*}
$ R_n = \frac{1}{n!}\left\{ \sum_{i=0}^{n} {}_n C_r \ppdrv{^n f}{x^i }{ y^{n-i}} (a', b') (x-a)^{i}(y-b)^{n-i}\right\}$を\underline{剰余項}という.

特に剰余項について, $\lim_{n \rightarrow \infty} R_n = 0$のとき, 
  \begin{align*}
  \begin{split}
  f(x,y) &= f(a,b) + \pdrv{f}{x}(a,b)(x-a) + \pdrv{f}{y}(a,b)(y-b) \\
  &+ \frac{1}{2} \left\{  \pdrv{^2f}{x^2}(a,b)(x-a)^2 +2 \ppdrv{^2f}{x}{y}(a,b)(x-a)(y-b)+
   \pdrv{^2f}{y^2}(a,b)(y-b) ^2    \right\}\\
   &+ \cdots \\
   &+ \frac{1}{n!}\left\{ \sum_{i=0}^{n} {}_n C_r \ppdrv{^n f}{x^i }{ y^{n-i}} (a, b) (x-a)^{i}(y-b)^{n-i}\right\} \\
   &+ \cdots .
    \end{split}
  \end{align*}
    \end{thm}
    \end{tcolorbox}

\begin{exa}
$f(x,y) = e^{x+y}$とする.
$\ppdrv{^n f}{x^i }{ y^{n-i}}(0,0) =1 $であり$\lim_{n \rightarrow \infty} R_n = 0$より
$$
 e^{x+y} = 1 + x+ y 
  + \frac{1}{2} \left (x^2 + 2xy +y^2    \right)
   + \cdots 
   + \frac{1}{n!} \left (x +y    \right)^n
  + \cdots .
$$
\end{exa}


 
\end{document}
