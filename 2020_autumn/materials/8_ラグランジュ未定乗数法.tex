\documentclass[dvipdfmx,a4paper,11pt]{article}
\usepackage[utf8]{inputenc}
%\usepackage[dvipdfmx]{hyperref} %リンクを有効にする
\usepackage{url} %同上
\usepackage{amsmath,amssymb} %もちろん
\usepackage{amsfonts,amsthm,mathtools} %もちろん
\usepackage{braket,physics} %あると便利なやつ
\usepackage{bm} %ラプラシアンで使った
\usepackage[top=30truemm,bottom=30truemm,left=25truemm,right=25truemm]{geometry} %余白設定
\usepackage{latexsym} %ごくたまに必要になる
\renewcommand{\kanjifamilydefault}{\gtdefault}
\usepackage{otf} %宗教上の理由でmin10が嫌いなので


\usepackage[all]{xy}
\usepackage{amsthm,amsmath,amssymb,comment}
\usepackage{amsmath}    % \UTF{00E6}\UTF{0095}°\UTF{00E5}\UTF{00AD}\UTF{00A6}\UTF{00E7}\UTF{0094}¨
\usepackage{amssymb}  
\usepackage{color}
\usepackage{amscd}
\usepackage{amsthm}  
\usepackage{wrapfig}
\usepackage{comment}	
\usepackage{graphicx}
\usepackage{setspace}
\setstretch{1.2}


\newcommand{\R}{\mathbb{R}}
\newcommand{\Z}{\mathbb{Z}}
\newcommand{\N}{\mathbb{N}}
\newcommand{\C}{\mathbb{C}} 



   %当然のようにやる.
\allowdisplaybreaks[4]
   %もちろん.
%\title{第1回. 多変数の連続写像 (岩井雅崇, 2020/10/06)}
%\author{岩井雅崇}
%\date{2020/10/06}
%ここまで今回の記事関係ない
\usepackage{tcolorbox}
\tcbuselibrary{breakable, skins, theorems}

\theoremstyle{definition}
\newtheorem{thm}{定理}
\newtheorem{lem}[thm]{補題}
\newtheorem{prop}[thm]{命題}
\newtheorem{cor}[thm]{系}
\newtheorem{claim}[thm]{主張}
\newtheorem{dfn}[thm]{定義}
\newtheorem{rem}[thm]{注意}
\newtheorem{exa}[thm]{例}
\newtheorem{conj}[thm]{予想}
\newtheorem{prob}[thm]{問題}
\newtheorem{rema}[thm]{補足}

\DeclareMathOperator{\Ric}{Ric}
\DeclareMathOperator{\Vol}{Vol}
 \newcommand{\pdrv}[2]{\frac{\partial #1}{\partial #2}}
 \newcommand{\drv}[2]{\frac{d #1}{d#2}}
  \newcommand{\ppdrv}[3]{\frac{\partial #1}{\partial #2 \partial #3}}


%ここから本文.
\begin{document}
%\maketitle
\begin{center}
{\Large 第8回. ラグランジュ未定乗数法 (川平先生の本, 第24章の内容)}
\end{center}

\begin{flushright}
 岩井雅崇, 2020/12/01
\end{flushright}

\section{ラグランジュ未定乗数法}
 
 \begin{tcolorbox}[
    colback = white,
    colframe = green!35!black,
    fonttitle = \bfseries,
    breakable = true]
    \begin{thm}
    \label{lan}
    $f(x,y)$, $g(x,y)$を領域$D$上の$C^1$級関数とする.
    $g(x,y)=0$のもとで$f(x,y)$が点$(a,b)$で極値を持つとし, 
    $\left(\pdrv{g}{x}(a,b),  \pdrv{g}{y}(a,b)\right) \neq (0,0)$とする.
    
    このとき,  ある定数$\lambda$があって
    $$
    \pdrv{f}{x}(a,b) = \lambda \pdrv{g}{x}(a,b), \pdrv{f}{y}(a,b) = \lambda \pdrv{g}{y}(a,b) \text{\,\,\,となる.}
    $$
    \end{thm}
    \end{tcolorbox}
    
 上の定理\ref{lan}から$F(x,y,t) = f(x,y)-tg(x,y)$とするとき, 
 $g(x,y)=0$のもとでの$f(x,y)$の極値の候補は以下の2つである.
 \begin{enumerate}
  \item $g(a,b)=\pdrv{g}{x}(a,b)=\pdrv{g}{y}(a,b)=0$となる点$(a,b)$.
 \item ある$\lambda$があって
 $\pdrv{F}{x}(a,b,\lambda) = \pdrv{F}{y}(a,b,\lambda) = \pdrv{F}{t}(a,b,\lambda)=0$となる点$(a,b)$.
 \end{enumerate}




\section{ラグランジュ未定乗数法の使い方}  
 \begin{tcolorbox}[
    colback = white,
    colframe = green!35!black,
    fonttitle = \bfseries,
    breakable = true]

 $g(x,y)=0$のもとで$f(x,y)$の極値を求める手順は以下の通りである.
 
 \begin{description}


 \item{[手順1.]} $g(a,b)=\pdrv{g}{x}(a,b)=\pdrv{g}{y}(a,b)=0$となる点$(a,b)$を求める.
 
   \item{[手順2.]} $F(x,y,t) = f(x,y)-tg(x,y)$とおいて, 
   $\pdrv{F}{x}(a,b,\lambda) = \pdrv{F}{y}(a,b,\lambda) = \pdrv{F}{t}(a,b,\lambda)=0$となる点$(a,b,\lambda)$を求める. %\footnote{$\lambda$の値は以後使わないので, 具体的に計算する必要はあまりない.}
   
 \item{[手順3.]} 手順1,手順2で求めた点$(a,b)$について, その値が極値であるかどうか調べる.
一般的な方法はないが, 例\ref{lan_exa}のように「最大値の存在」と「最大値, 最小値であれば極値である」ことを用いる方法もある.
 \end{description}

    \end{tcolorbox}
    
\begin{exa}
\label{lan_exa}
$f(x,y) = xy, g(x,y) = x^2+y^2-1$とする.
$g(x,y) =0$のもとでの$f(x,y)$の極値を求めよ.

つまり$S = \{ (x,y) \in \R^2: g(x,y)=0\}$とするとき, $f$の$S$上での極値を求めよ.

(解.) 上の手順通りに求める.

 [手順1.]
 $\pdrv{g}{x}=2x, \pdrv{g}{y}=2y$より, $g(a,b)=\pdrv{g}{x}(a,b)=\pdrv{g}{y}(a,b)=0$となる点は存在しない.
 
[手順2.] $F(x,y,t) = f(x,y)-tg(x,y) = xy - t(x^2 + y^2 -1)$とおく.
以下の方程式を解く.
$$
\pdrv{F}{x} = y-2xt=0,
\pdrv{F}{y}= x-2yt=0,
\pdrv{F}{t} =-(x^2+y^2-1)=0.
$$
すると$(x,y) =\pm \left(\frac{1}{\sqrt{2}},\frac{1}{\sqrt{2}}\right) , \pm \left(\frac{1}{\sqrt{2}},-\frac{1}{\sqrt{2}}\right)$
の4点が極値の候補となる.

[手順3.] $S $は有界閉集合より, $f$は$S$上で連続であるため, 第1回でやった定理より, 
$f$は$S$上で最大値・最小値を持つ.
よって$\pm \left(\frac{1}{\sqrt{2}},\frac{1}{\sqrt{2}}\right) , \pm \left(\frac{1}{\sqrt{2}},-\frac{1}{\sqrt{2}}\right)$の中に最大値をとる点や最小値をとる点がある.

実際計算すると, 
$$
f\left(\pm \left(\frac{1}{\sqrt{2}},\frac{1}{\sqrt{2}}\right)\right)=\frac{1}{2}, 
f\left(\pm \left(\frac{1}{\sqrt{2}},-\frac{1}{\sqrt{2}}\right)\right)=-\frac{1}{2}, 
$$
であるため, $\pm \left(\frac{1}{\sqrt{2}},\frac{1}{\sqrt{2}}\right) $で$f$は極大値(最大値)$\frac{1}{2}$をとり, 
$\pm \left(\frac{1}{\sqrt{2}},-\frac{1}{\sqrt{2}}\right)$で$f$は極小値(最小値)$-\frac{1}{2}$をとる.

\end{exa}

\section{ラグランジュ未定乗数法 3変数の場合}

 \begin{tcolorbox}[
    colback = white,
    colframe = green!35!black,
    fonttitle = \bfseries,
    breakable = true]
    \begin{thm}
    \label{lan}
    $f(x,y,z)$, $g(x,y,z)$を領域$D \subset \R^3$上の$C^1$級関数とし, 
    $F(x,y,t) = f(x,y,z)-tg(x,y,z)$とおく. 
    $g(x,y,z)=0$のもとで$f(x,y,z)$が点$(a,b,c)$で極値を持つとし, 
    $\left(\pdrv{g}{x}(a,b,c),  \pdrv{g}{y}(a,b,c), \pdrv{g}{z}(a,b,c)\right) \neq (0,0,0)$とする.
    
    このとき,  ある定数$\lambda$があって,
    $$
    \pdrv{F}{x}(a,b,c,\lambda) =\pdrv{F}{y}(a,b,c,\lambda) =\pdrv{F}{z}(a,b,c,\lambda) =\pdrv{F}{t}(a,b,c,\lambda) =0
     \text{\,\,\,となる.}
    $$
    \end{thm}
    \end{tcolorbox}
    
  \begin{exa}
  $f(x,y,z)= xyz, g(x,y,z)=x+y+z-170$とする.
  $g(x,y,z)=0$のもとで$f$の最大値を求めよ.
  つまり$S= \{ (x,y,z) \in \R^3: g(x,y,z)=0\}$とするとき, $f$の$S$上での最大値を求めよ.
  ただし$f$が$S$上で最大値を持つことは認めて良い.
  
  
  (解.) 手順通りに求める.
  
  [手順1.]
   $\pdrv{g}{x}=1$より$g(a,b,c)=\pdrv{g}{x}(a,b,c)=\pdrv{g}{y}(a,b,c)=\pdrv{g}{z}(a,b,c)=0$となる点$(a,b,c)$は存在しない.
   
[手順2.]
    $F(x,y,t) = f(x,y,z)-tg(x,y,z) = xyz-t(x+y+z-170)$とする.
   以下の方程式を解く.
$$
\pdrv{F}{x} = yz-t=0,
\pdrv{F}{y}= xz-t =0,
\pdrv{F}{z}=xy-t=0,
\pdrv{F}{t} =-(x+y+z-170)=0.
$$ 
すると$(x,y,z) =(170,0,0), (0,170,0), (0,170,0), (\frac{170}{3},\frac{170}{3},\frac{170}{3})$
の4点が極値の候補となる.

[手順3.] 最大値が存在し, 最大値は極値であるため, 上の4点の中に最大値をとる点が存在する.
実際計算すると, 
$$f(170,0,0)=0, f(0,170,0)=0, f(0,170,0)=0, f\left(\frac{170}{3},\frac{170}{3},\frac{170}{3}\right)=\left(\frac{170}{3}\right)^3
$$
であるため, $\left(\frac{170}{3},\frac{170}{3},\frac{170}{3}\right)$で$f$は最大値$\left(\frac{170}{3}\right)^3$をとる.
  \end{exa}


 

\end{document}
