\documentclass[dvipdfmx,a4paper,11pt]{article}
\usepackage[utf8]{inputenc}
%\usepackage[dvipdfmx]{hyperref} %リンクを有効にする
\usepackage{url} %同上
\usepackage{amsmath,amssymb} %もちろん
\usepackage{amsfonts,amsthm,mathtools} %もちろん
\usepackage{braket,physics} %あると便利なやつ
\usepackage{bm} %ラプラシアンで使った
\usepackage[top=30truemm,bottom=30truemm,left=25truemm,right=25truemm]{geometry} %余白設定
\usepackage{latexsym} %ごくたまに必要になる
\renewcommand{\kanjifamilydefault}{\gtdefault}
\usepackage{otf} %宗教上の理由でmin10が嫌いなので


\usepackage[all]{xy}
\usepackage{amsthm,amsmath,amssymb,comment}
\usepackage{amsmath}    % \UTF{00E6}\UTF{0095}°\UTF{00E5}\UTF{00AD}\UTF{00A6}\UTF{00E7}\UTF{0094}¨
\usepackage{amssymb}  
\usepackage{color}
\usepackage{amscd}
\usepackage{amsthm}  
\usepackage{wrapfig}
\usepackage{comment}	
\usepackage{graphicx}
\usepackage{setspace}
\setstretch{1.2}


\newcommand{\R}{\mathbb{R}}
\newcommand{\Z}{\mathbb{Z}}
\newcommand{\N}{\mathbb{N}}
\newcommand{\C}{\mathbb{C}} 



   %当然のようにやる.
\allowdisplaybreaks[4]
   %もちろん.
%\title{第1回. 多変数の連続写像 (岩井雅崇, 2020/10/06)}
%\author{岩井雅崇}
%\date{2020/10/06}
%ここまで今回の記事関係ない
\usepackage{tcolorbox}
\tcbuselibrary{breakable, skins, theorems}

\theoremstyle{definition}
\newtheorem{thm}{定理}
\newtheorem{lem}[thm]{補題}
\newtheorem{prop}[thm]{命題}
\newtheorem{cor}[thm]{系}
\newtheorem{claim}[thm]{主張}
\newtheorem{dfn}[thm]{定義}
\newtheorem{rem}[thm]{注意}
\newtheorem{exa}[thm]{例}
\newtheorem{conj}[thm]{予想}
\newtheorem{prob}[thm]{問題}
\newtheorem{rema}[thm]{補足}

\DeclareMathOperator{\Ric}{Ric}
\DeclareMathOperator{\Vol}{Vol}
 \newcommand{\pdrv}[2]{\frac{\partial #1}{\partial #2}}



%ここから本文.
\begin{document}
%\maketitle
\begin{center}
{\Large 第2回. 多変数関数の微分 (川平先生の本, 第17・18章の内容)}
\end{center}

\begin{flushright}
 岩井雅崇, 2020/10/13
\end{flushright}



\section{全微分}
\begin{tcolorbox}[
    colback = white,
    colframe = green!35!black,
    fonttitle = \bfseries,
    breakable = true]
    \begin{dfn}
    \label{total}
   \underline{ 関数$f(x,y)$が$(a,b)$で全微分可能}とは, ある定数$A,B$があって
    $$ E(x,y) = f(x,y)-\{  f(a,b)+ A(x-a) + B(y-b)\} \text{\,\, とするとき, }$$
      $$ \lim_{(x,y) \rightarrow (a,b)}\frac{|E(x,y)|}{\sqrt{(x-a)^2 + (y-b)^2}} =0 \text{\,\,となること.}$$
      
      $z = f(a,b) + A(x-a) + B(y-b)$を\underline{$f(x,y)$の点$(a,b)$での接平面の方程式}といい, その3次元グラフを\underline{接平面}という.
      
      $f$が領域$D$の任意の点で全微分可能であるとき, \underline{$f$は$D$上で全微分可能}であるという.
     
    \end{dfn}
\end{tcolorbox}


\begin{exa}
$f(x,y) = -(x^2+ y^2)$は点$(0,0)$で全微分可能.

(証.) $A=B=0$とする.
$E(x,y) = -(x^2 + y^2) - \{ 0 +0(x-0) + 0(y-0)\}= -(x^2 + y^2) $より, 
$$\lim_{(x,y) \rightarrow (0,0)}\frac{|E(x,y)|}{\sqrt{x^2 + y^2}} = \lim_{(x,y) \rightarrow (0,0)} \sqrt{x^2 + y^2} =0.$$
よって全微分可能.

接平面の方程式は
$$ z = 0 + 0(x-0) + 0(y-0) =0 $$
\end{exa}

\section{偏微分}

\begin{tcolorbox}[
    colback = white,
    colframe = green!35!black,
    fonttitle = \bfseries,
    breakable = true]
    \begin{dfn}
    \label{partial}
    \underline{関数$f(x,y)$が$(a,b)$で偏微分可能}とは, 2つの極限
    $$A = \lim_{x \rightarrow a} \frac{f(x,b) - f(a,b)}{x-a} \text{,\,\, \,\,} 
    B = \lim_{y \rightarrow b}\frac{f(a,y) - f(a,b)}{y-b}  \text{\,\,が存在すること.\,\,} $$
    $A,B$を\underline{$f(x,y)$の$(a,b)$での偏微分係数}と呼び, 
    $$A=\pdrv{f}{x}(a,b)    \text{,\,\, \,\,} B=\pdrv{f}{y}(a,b)\text{\,\,とかく.\,\,}$$

$f$が領域$D$の任意の点で偏微分可能であるとき, \underline{$f$は$D$上で偏微分可能}であるという.
     
    \end{dfn}
\end{tcolorbox}
$\pdrv{f}{x}(a,b) $は$\pdrv{f}{x}|_{(x,y)=(a,b)}$とかくこともある.


\begin{tcolorbox}[
    colback = white,
    colframe = green!35!black,
    fonttitle = \bfseries,
    breakable = true]
    \begin{dfn}
    $D$上で偏微分可能な関数$f$について
    
   $$
\begin{array}{ccccccccc}
\pdrv{f}{x}: &D & \rightarrow & \R & &\pdrv{f}{y}: &D & \rightarrow & \R \\
&(x,y) & \longmapsto & \pdrv{f}{x}(x,y)& & &(x,y) & \longmapsto & \pdrv{f}{y}(x,y)
\end{array}
$$
    
を\underline{$f(x,y)$の偏導関数}という.
     
    \end{dfn}
    
\end{tcolorbox}


\begin{exa}
\begin{itemize}

\item $f(x,y) = x^2y^3$は$\R^2$で偏微分可能.
偏導関数は
$\pdrv{f}{x} = 2xy^3, \pdrv{f}{y} = 3x^2y^2 $である.
\item $f(x,y) = \sqrt{1-x^2-y^2}$は$\{ (x,y) \in \R^2 : \sqrt{x^2 + y^2} <1 \}$上で偏微分可能.
偏導関数は
$$\pdrv{f}{x} = \frac{-x}{\sqrt{1-x^2-y^2}}\text{,\,\,\,\,}
\pdrv{f}{y} = \frac{-y}{\sqrt{1-x^2-y^2}}\text{\,\,である.\,\,}$$
\end{itemize}
\end{exa}

\begin{tcolorbox}[
    colback = white,
    colframe = green!35!black,
    fonttitle = \bfseries,
    breakable = true]
    \begin{dfn}
    
    $f$を領域$D$上の関数とする.
    $f$が$D$上で偏微分可能であり, その偏導関数$\pdrv{f}{x}, \pdrv{f}{y} $が$D$上で連続であるとき, \underline{$f$は$C^1$級である}という.
     
    \end{dfn}
    \end{tcolorbox}
\begin{exa}
 $f(x,y) = x^2y^3$は$C^1$級である.
 (みんながよく知っている関数は$C^1$級関数.)
\end{exa}

\section{全微分, 偏微分, $C^1$級の関係}

\begin{tcolorbox}[
    colback = white,
    colframe = green!35!black,
    fonttitle = \bfseries,
    breakable = true]
    \begin{thm}
    
    $f$を領域$D$上の関数とする.
    $f$が$D$上で$C^1$級ならば全微分可能である.
    
    特に$D$上で$C^1$級な関数$f$と$(a,b) \in D$において, $A=\pdrv{f}{x}(a,b)$, $B=\pdrv{f}{y}(a,b)$とするとき, 
    $$\lim_{(x,y) \rightarrow (a,b)}\frac{|E(x,y)|}{\sqrt{(x-a)^2 + (y-b)^2}} =0.$$
    % \footnote{定義\ref{total}に同じ.$E(x,y) := f(x,y)-\{  f(a,b)+ A(x-a) + B(y-b).\}$ }
    ここで$E(x,y) = f(x,y)-\{  f(a,b)+ A(x-a) + B(y-b)\}$とする.(定義\ref{total}と同様.)
    \end{thm}
    \end{tcolorbox}
    
    
    
\begin{tcolorbox}[
    colback = white,
    colframe = green!35!black,
    fonttitle = \bfseries,
    breakable = true]
    \begin{thm}
    
    $f$を領域$D$上の関数とする.
    $f$が$D$上で全微分可能なら, 偏微分可能である.
    
    特に定義\ref{total}の状況下において, $(a,b) \in D$について, $A=\pdrv{f}{x}(a,b)$, $B=\pdrv{f}{y}(a,b)$である.
    \end{thm}
    \end{tcolorbox}

\begin{tcolorbox}[
    colback = white,
    colframe = green!35!black,
    fonttitle = \bfseries,
    breakable = true]
    \begin{thm}
    \label{totalconti}
    $f$を領域$D$上の関数とする.
    $f$が$D$上で全微分可能なら, 連続である.
    \end{thm}
    \end{tcolorbox}
    
\begin{exa}
$f(x,y) = -(x^2+ y^2)$は$C^1$級関数. よって全微分可能.
点$(0,0)$での偏微分係数は
$A=\pdrv{f}{x}(0,0) =0$, $B=\pdrv{f}{y}(0,0) =0$.
接平面の方程式は
$z=0+A(x-0)+B(y-0)=0$.

 
\end{exa}

\begin{rem}
「全微分可能だが$C^1$級でない関数」, 「偏微分可能だが全微分可能でない関数」, 「連続だが全微分可能でない関数」, 「連続だが偏微分可能でない関数」. 「偏微分可能だが連続でない関数」などなど, いろいろな例がある.
\end{rem}

\begin{exa}偏微分可能だが全微分可能でない関数の例.

$$
  f(x,y)= \begin{cases}
     0& x\neq0 \text{\,かつ\,} y\neq0\\
    1& x=0 \text{\,または\,} y=0
  \end{cases}
  $$
 $f$は$(0,0)$で偏微分可能. 
 $\lim_{x \rightarrow 0} \frac{f(x,0) - f(0,0)}{x-0} =   \lim_{x \rightarrow 0} \frac{1-1}{x} =0$より定義\ref{partial}の極限が存在するから. 
 
 しかし$f$は$(0,0)$で全微分可能ではない.
 もし全微分可能ならば
 $$E(x,y)=f(x,y)-\{  f(0,0)+ 0(x-0) + 0(y-0)\}=f(x,y)-1$$とすると, 
 $\lim_{(x,y) \rightarrow (0,0)}\frac{|E(x,y)|}{\sqrt{x^2 + y^2}} =0$となる.
 よって$\lim_{(x,y) \rightarrow (0,0)} f(x,y)=1$となるが, 
 これは$(t,t)\rightarrow (0,0)$の$f$の極限を考えると矛盾である.
 \footnote{$f$は$(0,0)$で連続ではないからでも言える. (もし全微分可能ならば定理\ref{totalconti}より$f$は$(0,0)$で連続でないといけない.) }
\end{exa}

%$\pdrv{f}{x}$

\end{document}
