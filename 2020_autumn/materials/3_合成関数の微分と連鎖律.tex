\documentclass[dvipdfmx,a4paper,11pt]{article}
\usepackage[utf8]{inputenc}
%\usepackage[dvipdfmx]{hyperref} %リンクを有効にする
\usepackage{url} %同上
\usepackage{amsmath,amssymb} %もちろん
\usepackage{amsfonts,amsthm,mathtools} %もちろん
\usepackage{braket,physics} %あると便利なやつ
\usepackage{bm} %ラプラシアンで使った
\usepackage[top=30truemm,bottom=30truemm,left=25truemm,right=25truemm]{geometry} %余白設定
\usepackage{latexsym} %ごくたまに必要になる
\renewcommand{\kanjifamilydefault}{\gtdefault}
\usepackage{otf} %宗教上の理由でmin10が嫌いなので


\usepackage[all]{xy}
\usepackage{amsthm,amsmath,amssymb,comment}
\usepackage{amsmath}    % \UTF{00E6}\UTF{0095}°\UTF{00E5}\UTF{00AD}\UTF{00A6}\UTF{00E7}\UTF{0094}¨
\usepackage{amssymb}  
\usepackage{color}
\usepackage{amscd}
\usepackage{amsthm}  
\usepackage{wrapfig}
\usepackage{comment}	
\usepackage{graphicx}
\usepackage{setspace}
\setstretch{1.2}


\newcommand{\R}{\mathbb{R}}
\newcommand{\Z}{\mathbb{Z}}
\newcommand{\N}{\mathbb{N}}
\newcommand{\C}{\mathbb{C}} 



   %当然のようにやる.
\allowdisplaybreaks[4]
   %もちろん.
%\title{第1回. 多変数の連続写像 (岩井雅崇, 2020/10/06)}
%\author{岩井雅崇}
%\date{2020/10/06}
%ここまで今回の記事関係ない
\usepackage{tcolorbox}
\tcbuselibrary{breakable, skins, theorems}

\theoremstyle{definition}
\newtheorem{thm}{定理}
\newtheorem{lem}[thm]{補題}
\newtheorem{prop}[thm]{命題}
\newtheorem{cor}[thm]{系}
\newtheorem{claim}[thm]{主張}
\newtheorem{dfn}[thm]{定義}
\newtheorem{rem}[thm]{注意}
\newtheorem{exa}[thm]{例}
\newtheorem{conj}[thm]{予想}
\newtheorem{prob}[thm]{問題}
\newtheorem{rema}[thm]{補足}

\DeclareMathOperator{\Ric}{Ric}
\DeclareMathOperator{\Vol}{Vol}
 \newcommand{\pdrv}[2]{\frac{\partial #1}{\partial #2}}
 \newcommand{\drv}[2]{\frac{d #1}{d#2}}


%ここから本文.
\begin{document}
%\maketitle
\begin{center}
{\Large 第3回. 合成関数の微分と連鎖律 (川平先生の本, 第19・20・21章の内容)}
\end{center}

\begin{flushright}
 岩井雅崇, 2020/10/20
\end{flushright}



%\section{連鎖率 part1}

\begin{tcolorbox}[
    colback = white,
    colframe = green!35!black,
    fonttitle = \bfseries,
    breakable = true]
    \begin{thm}
    
    $f(x,y)$を領域$D$上の$C^1$級関数とする.
    $x=x(t)$, $y=y(t)$を$t$に関する$C^1$級関数とし, $z(t) = f(x(t) , y(t))$とするとき, 
    $$
    \drv{z}{t} = \pdrv{f}{x}\drv{x}{t} + \pdrv{f}{y}\drv{y}{t}.
    $$
    \end{thm}
    \end{tcolorbox}

\begin{exa}
$f(x,y) = 2x^3y$, $x(t) = \cos t$, $y(t) = \sin t$, 
 $z(t) = f(x(t) , y(t))$とする.
 このとき
 $$
 \pdrv{f}{x} = 6x^2y,  \pdrv{f}{y}=2x^3, \drv{x}{t}=-\sin t, \drv{y}{t}=\cos t\text{,\,\, より}
 $$
 \begin{align*}
 \begin{split}
     \drv{z}{t} & = \pdrv{f}{x}\drv{x}{t} + \pdrv{f}{y}\drv{y}{t}
     = 6\cos^2 t\sin t (-\sin t) + 2 \cos^3 t \cos t
     = - 6\cos^2 t\sin^2 t  + 2 \cos^4 t.
   \end{split}
 \end{align*}
 
\end{exa}


\begin{tcolorbox}[
    colback = white,
    colframe = green!35!black,
    fonttitle = \bfseries,
    breakable = true]
    \begin{dfn}
    
    領域$D$上の$C^1$級関数を$x(u,v)$, $y(u,v)$とする.
    
 $$
\begin{array}{ccccc}
\Phi: &D & \rightarrow & \R^2 & \\
&(u,v) & \longmapsto & (x(u,v),y(u,v))&
\end{array}
$$
を$C^1$級変数変換という.
    \end{dfn}
    \end{tcolorbox}


\begin{exa}
\begin{itemize}
\item $a,b,c,d$を定数とする.
$\Phi(u,v)  = (au+bv, cu+dv)$は$C^1$級変数変換である.
これを1次変換という.
\item $\Phi(u,v)  = (u \cos v, u \sin v)$も$C^1$級変数変換である.
これを極座標変換という.
\end{itemize}

\end{exa}


\begin{tcolorbox}[
    colback = white,
    colframe = green!35!black,
    fonttitle = \bfseries,
    breakable = true]
    \begin{thm}
 領域$D$上の$C^1$級変数変換を
 $$
\begin{array}{ccccc}
\Phi: &D & \rightarrow & \R^2 & \\
&(u,v) & \longmapsto & (x(u,v),y(u,v))&
\end{array}
$$
とし, 領域$E ( \subset \Phi(D))$上の$C^1$級関数を$f(x,y)$とする.

 領域$D$上の$C^1$級$g(u,v)$を
 $$
\begin{array}{ccccc}
g = f \circ \Phi: &D & \rightarrow & \R & \\
&(u,v) & \longmapsto & f(x(u,v),y(u,v))&
\end{array}
$$
で定めるとき, 各偏導関数は以下の通りになる.

    $$
    \pdrv{g}{u} = \pdrv{f}{x}\pdrv{x}{u} + \pdrv{f}{y}\pdrv{y}{u}
    \text{,\,\,\, \,\,\,\,\,\,\,}
     \pdrv{g}{v} = \pdrv{f}{x}\pdrv{x}{v} + \pdrv{f}{y}\pdrv{y}{v}.
    $$
    \end{thm}
    \end{tcolorbox}
行列の記法を用いると以下のようにかける.
$$
\left( \pdrv{g}{u}  , \pdrv{g}{v}\right) 
=
\left( \pdrv{f}{x} , \pdrv{f}{y}\right) 
\left(\begin{array}{cc} \pdrv{x}{u} & \pdrv{x}{v} \\ \pdrv{y}{u}& \pdrv{y}{v} \\ \end{array} \right).
$$
\begin{exa}
$f(x,y)$を$C^1$級関数とし,$C^1$級変数変換を$(x(u,v),y(u,v)) = (u \cos v, u \sin v)$とする.
$g(u,v) = f(x(u,v), y(u,v))$とするとき, $\pdrv{g}{u}, \pdrv{g}{v}$を
$\pdrv{f}{x},\pdrv{f}{y}$を用いてあらわせ.

(解.)
$$
\pdrv{x}{u}=\cos v,\text{\,\,} \pdrv{x}{v}= -u\sin v,\text{\,\,}  \pdrv{y}{u}=\sin v,\text{\,\,}  \pdrv{y}{v}= u \cos v, \text{\,\,\,より} 
$$
$$
   \pdrv{g}{u} = \pdrv{f}{x}\pdrv{x}{u} + \pdrv{f}{y}\pdrv{y}{u}=\cos v\pdrv{f}{x} + \sin v\pdrv{f}{y}.
$$

$$
  \pdrv{g}{v} = \pdrv{f}{x}\pdrv{x}{v} + \pdrv{f}{y}\pdrv{y}{v}
   =-u\sin v \pdrv{f}{x} + u \cos v\pdrv{f}{y}.
$$


\end{exa}


\end{document}
